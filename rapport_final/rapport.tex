\documentclass[12pt]{scrreprt} % Set font size to 12pt

\usepackage{changepage} % Add the changepage package for the addmargin environment
\usepackage{blindtext} % Generate lorem ipsums
\usepackage{helvet} % Arial font (Arial is proprietary, so we use Helvetica instead)
\renewcommand{\familydefault}{\sfdefault}
\usepackage[margin=2.5cm]{geometry} % Set margins to 2.5cm on all sides
\usepackage{graphicx} % Add the graphicx package for including images
\usepackage{atbegshi} % Package for placing elements at absolute positions
\usepackage[allcolors=blue]{hyperref} % Add the hyperref package for hyperlinking
\usepackage{xcolor} % Add the xcolor package for color support
\usepackage{tabularx} % Add the tabularx package for advanced tables

\makeatletter
\def\@cite#1#2{$^{\mbox{\scriptsize [#1\if@tempswa , #2\fi]}}$}
\makeatother

% CONSIGNES POUR LA REDACTION DU RAPPORT DE PROJET

% CRITERES D'EVALUATION :
% Critères liés à votre travail et vos compétences :
% - Le respect des objectifs fixés
% - Les compétences techniques et scientifiques
% - L'aptitude à rechercher de l'information
% - L'implication et l'esprit d'initiative
%
% Critères liés à votre rapport de projet
% - La structuration du document
% - L'orthographe et l'expression écrite
% - Le descriptif de la mission
% - La valeur du contenu scientifique et technique
% - La précision des résumés en français, en anglais et la bibliographie
%
% Critères liés à votre soutenance
% - La qualité de la structuration et de l'expression
% - La qualité du diaporama
% - Le dynamisme et l'interactivité
% - La qualité des réponses, l'esprit critique sur le travail réalisé, qualité de la conclusion et des éventuelles perspectives.

% RAPPORT
% Informations générales
%
% Le rapport doit être rédigé par vos soins, en respectant les normes
% de rédaction indiquées ci-dessous. Il sera évalué sur la forme
% (orthographe, grammaire, ponctuation, langage approprié, etc.) et le fond
% (organisation, méthode, argumentation, etc.).
% Il devra avoir une longueur comprise entre 20 et 25 pages (hors annexes)
% et sera à rendre au format PDF.
% L'ensemble des comptes rendus hebdomadaires d'avancement constituera une rubrique annexe de votre rapport.
% Votre document doit comporter une bibliographie permettant d'identifier et de retrouver les documents cités.
%
% Le rapport est à remettre le 9 avril 2024.
%
% Format attendu du document
% - Police Arial
% - Taille des polices : Texte (12), Sections/Titres (14, gras)
% - Numérotation des sections : numérotation scientifique (1, 1.1, 1.1.1)
% - Numérotation des pages : centrale en bas de page
% - Numérotation des tableaux et des figures
% - Marges : 2,5 cm en haut, en bas, à gauche et à droite
% - Résumé (français) et abstract (anglais) sur la dernière page de couverture
% - La première et la quatrième de couverture doivent respecter scrupuleusement le modèle téléchargeable ci-après
% - La mise en page du corps du rapport est libre.




\begin{document}

% add the 0.jpg image to the top-left of the first page

\begin{titlepage}
    \newgeometry{left=1cm, right=1cm, top=1cm, bottom=1cm}

    \noindent\includegraphics{0.jpg}
    \includegraphics{1.jpg}
    \hfill
    \includegraphics{2.jpg}
    \vspace{1cm}


    École Polytechnique de l'Université de Tours

    64, Avenue Jean

    Portalis 37200

    TOURS, FRANCE

    (33)2-47-36-14-14

    \href{http://www.polytech.univ-tours.fr}{\textcolor{red}{www.polytech.univ-tours.fr}}


    \vfill
    \begin{center}

        \Huge
        Parcours des écoles d'ingénieur Polytech
        \vspace{0.3cm}

        \textbf{Année 2023-2024}
        \vspace{0.5cm}


        \textbf{Projet Informatique : Labyrinthe}
    \end{center}
    \vfill

    % footer
    \begin{minipage}[t]{0.5\textwidth}
        \begin{flushleft}
            Étudiants
            \vspace{0.1cm}
            \hrule % horizontal
            \vspace{0.5cm}

            \textbf{Pacôme Renimel--Lamiré}
            \href{mailto:pacome.renimel--lamire@etu.univ-tours.fr}{\textcolor{blue}{pacome.renimel--lamire@etu.univ-tours.fr}}

            \textbf{Esteban Laurent}
            \href{mailto:esteban.laurent@etu.univ-tours.fr}{\textcolor{blue}{esteban.laurent@etu.univ-tours.fr}}
        \end{flushleft}
    \end{minipage}
    \begin{minipage}[t]{0.5\textwidth}
        \begin{flushright}
            Encadrant

            \vspace{0.1cm}
            \hrule % horizontal
            \vspace{0.5cm}

            \textbf{Christophe Lenté}

            \href{mailto:christophe.lente@univ-tours.fr}{\textcolor{blue}{christophe.lente@univ-tours.fr}}

        \end{flushright}
    \end{minipage}

\end{titlepage}
\restoregeometry


\chapter*{Avertissement}
\addcontentsline{toc}{chapter}{Avertissement} % Add the chapter to the table of contents


Ce document a été rédigé par Pacôme Renimel--Lamiré et Esteban Laurent susnommés les auteurs.

L’École Polytechnique de l’Université François Rabelais de Tours est représentée par Christophe Lenté susnommé le tuteur académique.

Par l’utilisation de ce modèle de document, l’ensemble des intervenants du projet acceptent
les conditions définies ci-après.

Les auteurs reconnaissent assumer l’entière responsabilité du contenu du document ainsi que
toutes suites judiciaires qui pourraient en découler du fait du non-respect des lois ou des droits
d’auteur.
Les auteurs attestent que les propos du document sont sincères et assument l’entière responsabilité de la véracité des propos.
Les auteurs attestent ne pas s’approprier le travail d’autrui et que le document ne contient
aucun plagiat.
Les auteurs attestent que le document ne contient aucun propos diffamatoire ou condamnable
devant la loi.
Les auteurs reconnaissent qu’ils ne peuvent diffuser ce document en partie ou en intégralité
sous quelque forme que ce soit sans l’accord préalable du tuteur académique et de l’entreprise.
Les auteurs autorisent l’école polytechnique de l’université François Rabelais de Tours à diffuser tout ou partie de ce document, sous quelque forme que ce soit, y compris après transformation en citant la source. Cette diffusion devra se faire gracieusement et être accompagnée du présent avertissement.

\newpage
\renewcommand{\contentsname}{Table des matières} % Change the title of the table of contents to "Sommaire"
\tableofcontents


\newpage
\chapter*{Introduction}
\addcontentsline{toc}{chapter}{Introduction} % Add the chapter to the table of contents



\chapter{Intitulé du chapitre 1}

Ici je peux rajouter des trucs et citer un texte de la bibliographie\cite{greenwade93}
\chapter{Intitulé du chapitre 2}

\chapter{Intitulé du chapitre 3}


\chapter*{Conclusion}
\addcontentsline{toc}{chapter}{Conclusion} % Add the chapter to the table of contents


\newpage % Start a new page for the bibliography
\renewcommand{\bibname}{Bibliographie} % Change the title of the bibliography section to "Références"

\bibliographystyle{unsrt} % Set the bibliography style. Change "plain" to the style you want to use.
\bibliography{references} % Include the bibliography file. Change "references" to the name of your .bib file.
\addcontentsline{toc}{chapter}{Bibliographie} % Add the chapter to the table of contents

\chapter*{Compte rendus hebdomadaires}
\addcontentsline{toc}{chapter}{Compte rendus hebdomadaires} % Add the chapter to the table of contents



\newpage

\begin{center}
    \Huge
    Projet Informatique : Labyrinthe
\end{center}

\section*{Résumé}

\blindtext

\section*{Mots-clés}

Labyrinthe, ajouter, d'autres, mots, clés

\section*{Abstract}
\blindtext

\section*{Keywords}

Labyrinth, add, other, keywords

\hrulefill

\vfill

\newcolumntype{Y}{>{\raggedleft\arraybackslash}X} % Redefine the X column type to align the text to the right

\begin{table}[h]
    \begin{tabularx}{\textwidth}{|X|Y|}
        \hline
        Encadrant académique              & Étudiants                               \\
        \textcolor{red}{Christophe Lenté} & \textcolor{red}{Pacôme Renimel--Lamiré} \\
                                          & \textcolor{red}{Esteban Laurent}        \\
        \hline
    \end{tabularx}
\end{table}



\end{document}